\documentclass[a4paper,12pt]{article}
\usepackage[dutch]{babel}
\usepackage[final]{graphicx}
\usepackage{url}
\usepackage{a4wide}

\usepackage{textcomp}
\usepackage[utf8]{inputenc}
\usepackage{dtklogos}
\usepackage{microtype}
\usepackage[T1]{fontenc}
\usepackage{lmodern}


\newcommand{\bs}{$\backslash$}

\begin{document}
\begin{center}
  \huge Informaticawerktuigen

  \Huge \LaTeX{}: tweede oefenzitting
\end{center}

\section{\LaTeX{} Beamer}
In de vorige oefenzitting over \LaTeX{} hebben we gezien hoe je met
\LaTeX{} artikels kan schrijven en met \BibTeX{} referenties toevoegen.
In deze oefenzitting zullen we tonen hoe je met \LaTeX{} ook
presentaties kan maken.

Merk op dat hoewel \LaTeX{} zelf ongeveer de de facto standaard is
voor het schrijven van wetenschappelijke artikels in
onderzoeksdomeinen zoals computerwetenschappen of wiskunde, dat Beamer
veel minder standaard is. Veel wetenschappers gebruiken software zoals
MS Office Powerpoint of OpenOffice Presenter die ook bij het gewone
publiek bekend zijn. Voor het groepswerk van dit vak verwachten wij wel
dat je met \LaTeX{} Beamer een presentatie maakt.

Voor presentaties in \LaTeX{} gebruiken we het pakket Beamer. Dit
pakket gebruik je op ongeveer dezelfde manier als \LaTeX{} zelf. Je
maakt een \texttt{.tex} bronbestand, dat je dan door
\texttt{pdflatex}\footnote{Of een tool zoals \texttt{latexmk} die
  \texttt{pdflatex} gebruikt} laat compileren
naar een PDF\@. Deze PDF kan je met een PDF viewer dan in volledig scherm
weergeven tijdens je presentatie.

Voor \LaTeX{} Beamer bestaat een uitstekende, maar lange
handleiding.\footnote{\url{http://www.ctan.org/tex-archive/macros/latex/contrib/beamer/doc/beameruserguide.pdf},
  hoofdstuk 3}
De handleiding legt niet alleen uit hoe je Beamer moet gebruiken, maar
geeft ook veel bruikbare tips over het maken van goede presentaties.

\section{De presentatie}
Open het bestand \texttt{bestanden/presentatie.tex}. Compileer het
bestand twee keren en bekijk het resultaat. Stel je PDF viewer in op
de modus ``volledig scherm'' en blader door de presentatie om een idee
te krijgen van hoe je een Beamer-presentatie gebruikt.

De tekst op Toledo is gebaseerd op een template uit de
Beamer-distributie. De template is bedoeld als een vertrekpunt voor
presentaties van ongeveer twintig minuten met een standaardstructuur.
Je ziet dat er een titelblad, overzicht, inleiding, enkele
inhoudelijke secties en een samenvatting voorzien zijn. De slides zijn
gevuld met algemene tips over het gebruik van Beamer en het maken van
presentaties (bv.\ ``use itemize a lot''). Na elke sectie wordt weer de
inhoudstafel getoond om aan te geven hoe ver de presentatie gevorderd
is.

Schakel de ``volledig scherm''-modus weer uit. Onderaan elke pagina vind
je enkele (lichtjes verborgen) knoppen, probeer het effect hiervan
uit.

Je ziet dat je met Beamer professioneel uitziende presentaties kan
maken. In de volgende sectie bekijken we hoe je dit alles zelf
gebruikt.

\section{De Preambule}
Bekijk de \LaTeX{}-broncode van de presentatie. Het eerste wat we zien
is het volgende \texttt{\bs{}documentclass} commando.
\begin{verbatim}
\documentclass{beamer}
\end{verbatim}

In tegenstelling tot de vorige oefenzitting beginnen we dus met de
\textbf{beamer} documentclass. Dit zorgt ervoor dat er een presentatie
zal gegenereerd worden. Vervolgens zien we het commando
\verb~\usetheme{Warsaw}~. Dit stelt een bepaalde stijl in voor de
slides in de presentatie. De Warsaw-stijl levert vrij gevulde pagina's
op, met bovenaan de slides telkens een overzicht van de secties en
onderaan de titel en de namen van de auteurs.

Verander de stijl naar \textbf{default} of \textbf{Bergen}, compileer
de presentatie en bekijk het effect. Je ziet dat je presentatie er
telkens ineens helemaal anders uitziet. In de beamer handleiding vind je
meer info over het gebruik van stijlen (``themes'').

Het volgende commando dat we tegenkomen in de presentatie is
\texttt{\bs{}usepackage[dutch]\allowbreak{}\{babel\}}. Dit is een commando dat we ook
gebruikten in artikels en zorgt ervoor dat Nederlandstalige
woordafbreking en dergelijke gebruikt wordt. Vervolgens worden de
titel, de auteurs en de datum (of de gelegenheid) van de presentatie
gedefinieerd.
\begin{verbatim}
\title{Titel van deze presentatie}
\author{Janneke \and Mieke}
\date{3de Internationale Conferentie over ABCD}
\end{verbatim}

Verzin een titel voor een presentatie over jullie groepswerk en vul
deze titel in. Vul de namen van de leden van je groepje in als auteurs. Je kunt
meerdere namen scheiden met \texttt{\textbackslash and}. Als date vul je in ``Presentatie
groepswerken Informaticawerktuigen''.

Na deze informatie vinden we een stuk code binnen een
\verb~\AtBeginSubsection[]~ commando. Dit hoef je niet te begrijpen,
maar zorgt ervoor dat er aan het begin van elke subsectie een
inhoudstafel wordt getoond met de huidige subsectie aangeduid, en als
titel ``Outline''.

\section{De eigenlijke presentatie}
Hierna komen we bij het \verb~\begin{document}~ commando, dat het
begin van het eigenlijke document aangeeft. Tussen
\verb~\begin{frame}~ en \verb~\end{frame}~ wordt telkens een slide
gedefinieerd. We zien dat er eerst een titelslide en vervolgens een
slide met de inhoudstafel gedefinieerd worden. Zoals je ziet in die
tweede slide kan je aan het \verb~\begin{frame}~ commando een
argument meegeven dat de titel van de slide bepaalt. Vervang de
titel ``Outline'' door ``Inhoud''.

Maak nu na de inhoudstafel een nieuwe slide aan met als titel ``Het
vak Informaticawerktuigen''. In de slide geef je een overzicht van een
aantal onderwerpen die al behandeld zijn in het vak (bijvoorbeeld:
\LaTeX{}, \BibTeX{}, Linux, etc.). Som deze onderwerpen op met
behulp van een \texttt{itemize} omgeving. Zoek desnoods op in de
Beamer handleiding hoe je dit gebruikt.\footnote{tip: zie sectie 3.8}

In de voorbeeldpresentatie heb je gezien dat je ervoor kan zorgen dat
niet alle tekst op de slide tegelijk verschijnt. Beginnende gebruikers
van presentatietools maken hier vaak overdreven gebruik van (wat er
snel voor zorgt dat je de aandacht van het publiek te veel afleidt),
maar soms is het wel nuttig. In deze slide willen we bijvoorbeeld
enkele dingen toevoegen aan het lijstje van de inhoud van het vak
Informaticawerktuigen die jullie nog niet gezien hebben. Voeg de tekst
``Programmeren in C'' en ``Leren presenteren'' toe aan
het lijstje en zorg ervoor dat deze eerst verborgen blijven op de
slide. De twee toevoegingen moeten wel tegelijk verschijnen. Zoek op
in de Beamer handleiding hoe je dit doet of kijk even verder in het
voorbeeldbestand hoe dit daar gedaan wordt.

Zoek op hoe je het \verb~\alert~ commando kan gebruiken in een slide.
Zorg ervoor dat \LaTeX{} Beamer voorkomt in bovenstaand lijstje en
zorg ervoor dat het eerst gewoon in het lijstje voorkomt, maar bij de
volgende stap in je presentatie in het rood komt te staan. Pas dan
mogen de nog niet geziene dingen (C en presenteren) verschijnen.

\section{Enkele basis-presentatietips}
Goede presentaties maken is niet eenvoudig. Beginners en zelfs
mensen met ervaring maken vaak basisfouten. De basisregel die je moet
in je achterhoofd houden is dat de presentatie voornamelijk dient als
een overzicht van wat je wil vertellen, zodat je publiek beter de
structuur kan volgen van wat je vertelt. Daarbovenop is een
presentatie handig om formules, tabellen en afbeeldingen te tonen
waarover je iets wilt vertellen. Maar in al te veel presentaties zie
je volgende fouten terugkomen:

\begin{description}
  \item[Te veel animaties] Animaties leiden snel de aandacht af van je
        publiek. Gebruik ze enkel als ze iets bijdragen aan wat je wilt
        vertellen. Ook tekst die verschijnt bij opeenvolgende stappen in je
        presentatie leidt snel de aandacht af. Maak hier enkel \textbf{in
          heel beperkte mate} gebruik van.
  \item[Te veel details] In een presentatie mag je over het algemeen
        nooit meer details tonen dan je mondeling uitlegt. Verwacht nooit
        dat je publiek iets gaat lezen op je slide dat je niet mondeling
        toelicht. Zorg ervoor dat je illustraties enkel de essentie tonen.
  \item[Te veel tekst] Gebruik nooit volzinnen in een presentatie. Maak
        je zinnen zo kort mogelijk en beperk je nog het liefst tot enkel
        kernwoorden. Gebruik kernachtige titels. Een algemene richtlijn is
        dat een slide ongeveer 20 \`a 40 woorden mag bevatten.
  \item[Te weinig kleur] Visuele aanwijzingen in je tekst zijn veel
        duidelijker voor je publiek dan tekst. Zet iets bijvoorbeeld
        tijdelijk in het rood als je er in je uitleg de aandacht op wil
        trekken. Gebruik veel kleur, opsommingen en dergelijke.
  \item[Te veel slides] Als algemene richtlijn mag je nooit meer dan
        \'e\'en slide per minuut presentatie hebben en liefst minder, met
        desnoods een uitzondering voor tussentijdse inhoudstafels en dergelijke.
        Presentaties die te veel over tijd gaan worden op bijvoorbeeld
        wetenschappelijke conferenties niet getolereerd en meestal koudweg
        afgebroken.
\end{description}

\section{Ten slotte}
Tijdens de rest van deze oefenzitting is het de bedoeling dat jullie
beginnen aan een presentatie voor jullie groepswerk. De tijd die
jullie zullen hebben is heel beperkt, dus denk op voorhand heel goed
na over welk deel van jullie inhoud jullie zullen vertellen.


\flushright
\input{addendum.tex}
\end{document}

%%% Local Variables:
%%% mode: latex
%%% TeX-master: t
%%% End:
